\section{The Path Integral Formalism}

By the laws of quantum mechanics, the probability for a point particle that is in state $x_L$ at time $t_a$ to be found in state $x_R$ at a later time $t_b$ is
\begin{equation} \label{qmtp}
  (x_R t_b | x_L t_a) = \bra{x_R}\hat{U}(t_b,t_a)\ket{x_L}.
\end{equation}
We divide the time interval between $t_a$ and $t_b$ into a large number $N$ of smaller intervals by introducing time steps $t_n$ with $t_n -t_{n-1} = \frac{t_b - t_a}{N} =:\epsilon$. We split up the time evolution operator accordingly and equation \ref{qmtp} becomes
\begin{equation} \label{qmtpn}
  (x_R t_b | x_L t_a) = \bra{x_R}\hat{U}(t_b,t_{N-1}) \cdots \hat{U}(t_{n+1},t_n)\cdot\hat{U}(t_n,t_{n-1}) \cdots \hat{U}(t_2,t_1)\cdot\hat{U}(t_1,t_a)\ket{x_L}.
\end{equation}
Now between each two steps we insert the identity operator $1 = \int_{-\infty}^{+\infty} dx_n \ket{x_n}\bra{x_n}$ and we get 
\begin{align} \label{tpi}
  (x_R t_b | x_L t_a) &=  \bra{x_R}\hat{U}(t_b,t_{N-1}) \int_{-\infty}^{+\infty} dx_{N-1} \ket{x_{N-1}}\bra{x_{N-1}} \hat{U}(t_{N-1},t_{N-2}) \cdots \ket{x_L} \\
                      &=  (\prod_{n=1}^{N-1} \int_{-\infty}^{+\infty} dx_{n}) \bra{x_R}\hat{U}(t_b,t_{N-1})  \ket{x_{N-1}}\bra{x_{N-1}} \hat{U}(t_{N-1},t_{N-2}) \cdots \ket{x_L} \\
                      &=  (\prod_{n=1}^{N-1} \int_{-\infty}^{+\infty} dx_{n})\prod_{n=1}^{N} (x_n t_n | x_{n-1} t_{n-1})
\end{align}
For the terms 
\begin{equation}\label{tpn}
  (x_n t_n | x_{n-1} t_{n-1}) = \bra{x_n} \hat{U}(t_{n},t_{n-1}) \ket{x_{n-1}} = \bra{x_n} e^{\frac{-i \epsilon \hat{H}}{\hbar}} \ket{x_{n-1}} 
\end{equation}
we write the Hamiltonian as
\begin{equation}\label{HTV}
  \hat{H}(p,x,t) = \hat{T(p)} + \hat{V(x)}
\end{equation}
and use that the commutator $[\epsilon \hat{T}, \epsilon \hat{V}]$ is of order $\epsilon^2$ and thus for our small $\epsilon$ can be neglected. This means (we can get this step by using the Baker–Campbell–Hausdorff formula):
\begin{align}\label{UTV}
  \hat{U}(t_{n},t_{n-1}) = e^{\frac{-i \epsilon \hat{H}}{\hbar}} = e^{\frac{-i \epsilon (\hat T + \hat V)}{\hbar}} \\
  = e^{\frac{-i \epsilon \hat{T}}{\hbar}} e^{\frac{-i \epsilon \hat{V}}{\hbar}} (1 + \mathcal{O} (\epsilon^2) ) \approx e^{\frac{-i \epsilon \hat{T}}{\hbar}} e^{\frac{-i \epsilon \hat{V}}{\hbar}} 
\end{align}
Again inserting $1 = \int_{-\infty}^{+\infty} dx \ket{x}\bra{x}$, the single steps read
\begin{equation}\label{tpntv}
  (x_n t_n | x_{n-1} t_{n-1}) = \bra{x_n} e^{\frac{-i \epsilon \hat{T}}{\hbar}} e^{\frac{-i \epsilon \hat{V}}{\hbar}}  \ket{x_{n-1}} = \int_{-\infty}^{+\infty} dx \bra{x_n} e^{\frac{-i \epsilon \hat{T}}{\hbar}} \ket{x} \bra{x} e^{\frac{-i \epsilon \hat{V}}{\hbar}}  \ket{x_{n-1}}.
\end{equation}
That the potential energy $V(x)$ only depends on the position $x$ means that the position eigenkets $\ket{x}$ are eigenkets of $\hat{V}$:
\begin{align}\label{Vx}
  \hat{V} \ket{x} &= V(x) \ket{x}  \\
  \bra{x} e^{\frac{-i \epsilon \hat{V}}{\hbar}}  \ket{x_{n-1}} &= e^{\frac{-i \epsilon V(x_n-1)}{\hbar}} \delta (x,x_{n-1})
\end{align}
Thus \ref{tpntv} becomes
\begin{equation}\label{tpni}
   (x_n t_n | x_{n-1} t_{n-1}) = e^{\frac{-i \epsilon V(x_n-1)}{\hbar}} \bra{x_n} e^{\frac{-i \epsilon \hat{T}}{\hbar}} \ket{x_{n-1}}
\end{equation}



