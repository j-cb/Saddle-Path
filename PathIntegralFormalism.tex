\section{The Path Integral Formalism}

By the laws of quantum mechanics, the probability for a point particle that is in state $x_1$ at time $t_a$ to be found in state $x_2$ at a later time $t_b$ is
\begin{equation} \label{qmtp}
  (x_2 t_b | x_1 t_a) = \bra{x_2}\hat{U}(t_b,t_a)\ket{x_1}.
\end{equation}
We divide the time interval between $t_a$ and $t_b$ into a large number $N$ of smaller intervals by introducing time steps $t_n$ with $t_n -t_{n-1} = \frac{t_b - t_a}{N} =:\epsilon$. We split up the time evolution operator accordingly and equation \ref{qmtp} becomes
\begin{equation} \label{qmtpn}
  (x_2 t_b | x_1 t_a) = \bra{x_2}\cdot\hat{U}(t_b,t_{N-1}) \cdots \hat{U}(t_{n+1},t_n)\cdot\hat{U}(t_n,t_{n-1}) \cdots \hat{U}(t_2,t_1)\cdot\hat{U}(t_1,t_a)\ket{x_1}.
\end{equation}
Now between each two steps we insert the identity operator $1 = \int_{-\infty}^{+\infty} dx_n \ket{x_n}\bra{x_n}$ and we get 
\begin{equation} \label{tpi}
  (x_2 t_b | x_1 t_a) = \bra{x_2}\cdot\hat{U}(t_b,t_{N-1}) \int_{-\infty}^{+\infty} dx_{N-1} \ket{x_{N-1}}\bra{x_{N-1}} \cdot \hat{U}(t_{N-1},t_{N-2}) \cdots \ket{x_1}.
\end{equation}
