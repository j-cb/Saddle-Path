\section{The Path Integral Formalism}

By the laws of quantum mechanics, the probability for a point particle that is in state $x_L$ at time $t_a$ to be found in state $x_R$ at a later time $t_b$ is
\begin{equation} \label{qmtp}
  (x_R t_b | x_L t_a) = \bra{x_R}\hat{U}(t_b,t_a)\ket{x_L}.
\end{equation}
We divide the time interval between $t_a$ and $t_b$ into a large number $N$ of smaller intervals by introducing time steps $t_n$ with $t_n -t_{n-1} = \frac{t_b - t_a}{N} =:\epsilon$. We split up the time evolution operator accordingly and equation \ref{qmtp} becomes
\begin{equation} \label{qmtpn}
  (x_R t_b | x_L t_a) = \bra{x_R}\hat{U}(t_b,t_{N-1}) \cdots \hat{U}(t_{n+1},t_n)\cdot\hat{U}(t_n,t_{n-1}) \cdots \hat{U}(t_2,t_1)\cdot\hat{U}(t_1,t_a)\ket{x_L}.
\end{equation}
Now between each two steps we insert the identity operator $1 = \int\limits_{-\infty}^{+\infty} dx_n \ket{x_n}\bra{x_n}$ and we get 
\begin{align}
  (x_R t_b | x_L t_a) &=  \bra{x_R}\hat{U}(t_b,t_{N-1}) \int\limits_{-\infty}^{+\infty} dx_{N-1} \ket{x_{N-1}}\bra{x_{N-1}} \hat{U}(t_{N-1},t_{N-2}) \cdots \ket{x_L} \\
                      &=  (\prod_{n=1}^{N-1} \int\limits_{-\infty}^{+\infty} dx_{n}) \bra{x_R}\hat{U}(t_b,t_{N-1})  \ket{x_{N-1}}\bra{x_{N-1}} \hat{U}(t_{N-1},t_{N-2}) \cdots \ket{x_L} \\
                      &=  (\prod_{n=1}^{N-1} \int\limits_{-\infty}^{+\infty} dx_{n})\prod_{n=1}^{N} (x_n t_n | x_{n-1} t_{n-1}) \label{tpi}
\end{align}
For the terms 
\begin{equation}\label{tpn}
  (x_n t_n | x_{n-1} t_{n-1}) = \bra{x_n} \hat{U}(t_{n},t_{n-1}) \ket{x_{n-1}} = \bra{x_n} e^{\frac{-i}{\hbar} \epsilon \hat{H}} \ket{x_{n-1}} 
\end{equation}
we write the Hamiltonian as
\begin{equation}\label{HTV}
  \hat{H}(p,x,t) = \hat{T(p)} + \hat{V(x)}
\end{equation}
and use that the commutator $[\epsilon \hat{T}, \epsilon \hat{V}]$ is of order $\epsilon^2$ and thus for our small $\epsilon$ can be neglected. This means (we can get this step by using the Baker–Campbell–Hausdorff formula):
\begin{align}\label{UTV}
  \hat{U}(t_{n},t_{n-1}) = e^{\frac{-i}{\hbar} \epsilon \hat{H}} = e^{\frac{-i}{\hbar} \epsilon (\hat T + \hat V)} \\
  = e^{\frac{-i}{\hbar} \epsilon \hat{T}} e^{\frac{-i}{\hbar} \epsilon \hat{V}} (1 + \mathcal{O} (\epsilon^2) ) \approx e^{\frac{-i}{\hbar} \epsilon \hat{T}} e^{\frac{-i}{\hbar} \epsilon \hat{V}} 
\end{align}
Again inserting $1 = \int\limits_{-\infty}^{+\infty} dx \ket{x}\bra{x}$, the single steps read
\begin{equation}\label{tpntv}
  (x_n t_n | x_{n-1} t_{n-1}) = \bra{x_n} e^{\frac{-i}{\hbar} \epsilon \hat{T}} e^{\frac{-i}{\hbar} \epsilon \hat{V}}  \ket{x_{n-1}} = \int\limits_{-\infty}^{+\infty} dx \bra{x_n} e^{\frac{-i}{\hbar} \epsilon \hat{T}} \ket{x} \bra{x} e^{\frac{-i \epsilon \hat{V}}{\hbar}}  \ket{x_{n-1}}.
\end{equation}
That the potential energy $V(x)$ only depends on the position $x$ means that the position eigenkets $\ket{x}$ are eigenkets of $\hat{V}$:
\begin{align}\label{Vx}
  \hat{V} \ket{x} &= V(x) \ket{x}  \\
  \bra{x} e^{\frac{-i}{\hbar} \epsilon \hat{V}}  \ket{x_{n-1}} &= e^{\frac{-i}{\hbar} \epsilon V(x_{n-1})} \delta (x,x_{n-1})
\end{align}
Thus \ref{tpntv} becomes
\begin{equation}\label{tpni}
   (x_n t_n | x_{n-1} t_{n-1}) = e^{\frac{-i \epsilon V(x_{n-1})}{\hbar}} \bra{x_n} e^{\frac{-i \epsilon \hat{T}(\hat{p})}{\hbar}} \ket{x_{n-1}}
\end{equation}
and as we want to evaluate the operator $\hat{T}$ which is describen in the momentum basis, we insert $1 = \int\limits_{-\infty}^{+\infty} dp \ket{p}\bra{p}$ and get
\begin{align}\label{tpnp}
   (x_n t_n | x_{n-1} t_{n-1})  &= e^{\frac{-i \epsilon V(x_{n-1})}{\hbar}} \bra{x_n} e^{\frac{-i \epsilon \hat{T}(\hat{p})}{\hbar}} \int\limits_{-\infty}^{+\infty} dp \ket{p}\braket{p|x_{n-1}} \\
                                &= \int\limits_{-\infty}^{+\infty} dp \ e^{\frac{-i \epsilon V(x_{n-1})}{\hbar}} \bra{x_n} e^{\frac{-i \epsilon \hat{T}(\hat{p})}{\hbar}} \ket{p} \frac{1}{\sqrt{2 \pi \hbar}} e^{\frac{i p x_{n-1}}{\hbar}} \\
                                &= \int\limits_{-\infty}^{+\infty} dp \ e^{\frac{-i \epsilon V(x_{n-1})}{\hbar}} e^{\frac{-i \epsilon T(p)}{\hbar}} \braket{x_n|p} \frac{1}{\sqrt{2 \pi \hbar}} e^{\frac{- i p x_{n-1}}{\hbar}} \\
                                &= \int\limits_{-\infty}^{+\infty} dp \ e^{\frac{-i \epsilon V(x_{n-1})}{\hbar}} e^{\frac{-i \epsilon T(p)}{\hbar}} \frac{1}{\sqrt{2 \pi \hbar}} e^{\frac{i p x_{n}}{\hbar}} \frac{1}{\sqrt{2 \pi \hbar}} e^{\frac{- i p x_{n-1}}{\hbar}}  \\
                                &= \int\limits_{-\infty}^{+\infty} \frac{dp}{2 \pi \hbar}  \ e^{\frac{-i \epsilon V(x_{n-1})}{\hbar}} e^{\frac{-i \epsilon T(p)}{\hbar}} e^{\frac{i p (x_{n} -x_{n-1})}{\hbar}}.
\end{align}
Now we can use that for very small $\epsilon = t_n - t_{n-1}$ the definition of the time derivative gives
\begin{equation}\label{deriv}
  x_{n} - x_{n-1} = \epsilon \frac{x_{n} -x_{n-1}}{t_n - t_{n-1}} =  \epsilon \dot{x}_n
\end{equation}
and we get
\begin{align}
   (x_n t_n | x_{n-1} t_{n-1})  &=  \int\limits_{-\infty}^{+\infty} \frac{dp}{2 \pi \hbar}  \ e^{\frac{-i \epsilon ( V(x_{n-1}) + T(p) - p \dot{x}_n)}{\hbar}}
                                &=  \int\limits_{-\infty}^{+\infty} \frac{dp}{2 \pi \hbar}  \ e^{\frac{i \epsilon (p \dot{x}_n - H(x_{n-1},p))}{\hbar}}.
\end{align}
We put this into \ref{tpi} and get 
\begin{align}
  (x_R t_b | x_L t_a) &=  (\prod_{n=1}^{N-1} \int\limits_{-\infty}^{+\infty} dx_{n})\prod_{n=1}^{N} (\int\limits_{-\infty}^{+\infty} \frac{dp_n}{2 \pi \hbar}  \ e^{\frac{i \epsilon (p_n \dot{x}_n - H(x_{n-1},p_n))}{\hbar}})  \\
                      &=  (\prod_{n=1}^{N-1} \int\limits_{-\infty}^{+\infty} dx_{n})(\prod_{n=1}^{N} \int\limits_{-\infty}^{+\infty}  \frac{dp_n}{2 \pi \hbar}) \ e^{\sum_{n=1}^{N}\frac{i \epsilon (p_n \dot{x}_n - H(x_{n-1},p_n))}{\hbar}}
\end{align}
